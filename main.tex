% This is samplepaper.tex, a sample paper demonstrating the
% LLNCS class package for the SMU Data Science Review Journal;
%
% This sample paper is a modified version of samplepaper.tex for
% the Springer Computer Science proceedings; Version 2.20 of 2017/10/04
%
% Version 1.0 2019/06/03

% Use the llncs.cls formatting
\documentclass{llncs}

% Set the packages for use within the document. The following 
% packages should be included.  Additional packages that do not
% conflict with these packages or change the llncs class formatting
% may be used.  Packages that do change the formatting are
% not allowed.
\usepackage{graphicx} % Used for displaying a sample figure. 
% If possible, figure files should be included in EPS format. 
% PDF format is also acceptable. JPEG  will work, but some of 
% them are downsampled resulting in fuzzy images.
\usepackage{booktabs} % Better horizontal rules in tables
\usepackage{multirow} % Better combined rows in tables

% The title of the paper
\title{MSDS 6120 Capstone A}

% The complete list of authors with their affiliations
\author{
Daniel Serna\inst{1} \and
James Vasquez\inst{1,2} \and
Donald Markley\inst{2}
}

% The Institutes and emails associated with each author. All students
% should use their MSDS affiliation or a generic SMU affiliation.
% Advisors should use their appropriate affiliation. Note that advisors
% are NOT referenced or otherwise denoted as advisors. Advisors
% are simply co-authors on the paper.
% Note that the emails for the MSDS affiliation, show how 
% to list emails that have the same organization portion.
\institute{
Master of Science in Data Science, Southern Methodist University,
Dallas TX 75275 USA 
\email{\{dserna,vasquezj\}@smu.edu} \and
Triple Crown Resources. ZipCode City, State
\email{corporatesponser@email.com} \\
\url{http://www.companywebsite.com}
}

% Begin the document
\begin{document}

\maketitle              % typeset the title and author of the paper

% Reset the footnote counter
\setcounter{footnote}{0}
% The abstract environment uses the \begin{} and \end{} constructs to 
% denote the beginning and ending of the abstract contents.
\begin{abstract}
%The abstract should briefly summarize the contents of the paper in
%150--250 words.  The abstract should contain approximately six sentences contained within a single paragraph. The first sentence typically begins with ``In this paper, we present'' which is then followed by what is presented in the paper.  The second sentence is used to motivate the importance of what is presented and define the broad problem domain. Then, two to three sentences are used to state how the problem was solved. A single statement of the main result (singular) is then followed by a single statement of the main conclusion (singular).
In this paper we present a means of predicting when a slide event will occur in horizontal drilling operations. Optimizing horizontal drill operations to reduce slide events yields significant benefit in terms of cost and time. We achieve this benefit by creating a predictive model based on a number of features present in time series data provided by Triple Crown Resources.

% Keywords may be used, but they are not required.
%\keywords{First keyword  \and Second keyword \and Another keyword.}
\end{abstract}

% Sections are denoted by the use of the \section{Section Name} 
% command -- where "Section Name" is the name you give to the Section.
\section{Introduction}

Drilling a well has many moving parts in order to reach total depth (TD) of the well. Triple Crown Resources (TCR) has requested to gain insight in an issue that has the potential to reduce the time to reach TD. While drilling in a horizontal fashion the sensors on the drill record the rate of penetration (ROP), the higher the ROP the faster it drill through the subsurface rock. A stable and higher ROP will allow the drilling operations reach TD faster. The current issue is that the drill reduces ROP or slides instead of rotating through the subsurface rock. The sliding causes a dramatic decrease in ROP, which ultimately takes longer to reach TD when dealing with slide events while drilling.

We will determine the probability of a slide event occuring within a specified time frame in horizontal drilling operations.  

TCR has access to all sensor data of 21 wells and growing. The data is captured at ever second while drilling, due to this the amount of data is very massive. It was the team decision to take samples of the data every 10 seconds, which reduced the amount of rows to 2.98 million rows of data. In addition to this the data contained 506 features.

Data reduction was the first step into limiting the data, using SpotFire a visualization tool the team removed columns that had NULL values, columns with only 1 number in it, and took a cut off of 90\% data filled in. These items reduced the features to 122 columns for initial analysis. Working with TCR it was indicated that rows of data could be limited to rows with an inclination 85 degrees or higher. This inclination indicates that this is the horizontal portion of the well and not the vertical section. This business knowledge reduced the initial data set to 725K rows of data.

% A second section is begun with another \section{} command
\section{History}
The Permian Basin stretches from lower the Southern portion of New Mexico and extends to much of West Texas. In geological terms the basin was formed during the Paleozoic Era which during this time range from shallow seas to vibrant oceans. From the geological timeline much of the structures which ultimately formed the traps for hydrocarbon were created during the late Paleozoic Era [Brit Refer] (251 million years ago).

First oil produced from the basin started in the middle of 1920’s, the first major activity during the basins life was during the 1950s [Global Refer]. Much of this data is still used today to deliver control points of the basin. These control points help geologist map different formations in the subsurface

Though the basin has been producing for more than five decades a technology that has evolved brought additional life into the basin. Hydraulic fracturing has been around since 1949 but it wasn’t till the early 2000’s that with a combination of horizontal drilling and fracking that an old technology brought a basin back to life. The ability to drill in a horizontal direction to stay within a formation gave an unprecedented way to drain reservoirs that were once thought to be on a decline for producing hydrocarbons.

A leap in technology and computer power leaped many industries including Energy into Big Data Analytics. However, it was the velocity of data being captured that complicated items [SD refer]. Specifically, when monitoring real time drilling sensors are capturing data every second.  Many in the industry were not experience in how to analyze and make the best use of this data.  With the market drop in 2014 it was first seen that Energy companies starting to treat data as an asset and put effort and recruit talent that could make the most out of data that has been sitting in 3rd party vendor databases.

%
% ---- Bibliography ----
%
% BibTeX users should specify bibliography style 'splncs04'.
% References will then be sorted according to alphabetical 
% and formatted in the correct style.
%
 \bibliographystyle{splncs04}
 \bibliography{samplebib}

% End the document
\end{document}
